%Elabore un reporte en el que presente las explicaciones que se ofrecen sobre las crisis económicas y en particular sobre la crisis mexicana. Comente las ideas que se presentan sobre los planes de estabilización y las características del Pacto de Solidaridad Económica (PSE) y del Pacto para la Estabilidad y el Crecimiento Económico (PECE). Comente la política de finanzas públicas instrumentada durante el PECE. pp. 13 - 37
Se dice que el periodo neoliberal en México comienza con la entrada de Miguel de la Madrid, no es de sorprender que por estas fechas también en Estados Unidos se halla presentado una ideología similar, desde mi percepción esta ideología se desarrolla a partir de los problemas económicos y financieros que sufrió el país, junto con un sistema económico que ya no funcionaba. Este nuevo periodo comienza como un nuevo paradigma para la estabilidad, dejando fuera las ideas de la administración pasada de aumentar el gasto público para contrarrestar la caída del PIB. Aquí se presenta un recuento de los antecedentes, la visión global y lo que se hizo para lograr la estabilización.


\section{Teoría del desarrollo de la crisis y caso de estudio}
Una crisis de recesión típica se caracteriza por una inhabilidad de pasar de una etapa de crecimiento estable a una de aceleración, principalmente por \textbf{inflexibilidad en el sistema financiero}.
Esto por las políticas restrictivas al mercado que en una etapa de aceleración deberían cambiarse a unas más librecambistas.
La recesión daña primero a la clase trabajadora porque el capital se mueve más fácilmente que el trabajo; a la vez la reducción de la producción genera inflación dañando aún más a la clase trabajadora.
La crisis en un país en desarrollo es diferente a la crisis en uno desarrollado, ya que el país en desarrollo depende muchas veces del mercado externo (América Latina principalmente) lo que en la crisis provoca una fuga de capitales, déficit y una espiral de inflación que dañan más al país, así se vuelven necesarias reservas internacionales o ahorro externo.
Para solucionar este problema se pueden llevar a cabo políticas ortodoxas o heterodoxas, las ortodoxas consideran la inflación como un fenómeno monetario, lo cual deja ver el tipo de política que implementa (solución de desequilibrios fiscales y externos); por otra parte, la política heterodoxo (la planteada por Aspe) es de corte estructuralista y considera que la inflación tiene efectos en la economía real y no es solo un fenómeno monetario.

Esta última deja ver las concepciones económicas de Ape y su base teórica para implementar la política económica de estabilización, que se resume en:
\begin{quote}
    Para alcanzar la estabilización no basta con corregir los desequilibrios fiscales o externos. Deben corregirse también las fuentes o causas de la inercia inflacionaria.
\end{quote}
Resulta interesante que Aspe menciona al sistema financiero como principal causa de recesión inflacionaria debido a la inhabilidad de seguir creciendo.

\subsection{Caso de estudio: Crisis mexicana del 82}
A partir de la 2da guerra mundial aparece en México y muchos otros países (Ver el New Deal) el estado proteccionista, el cual se hace cargo de la actividad económica, en México se ha este periodo se le conoce como el desarrollo estabilizador o el milagro mexicano. Su final llega con el desarrollo compartido que busca redistribuir la riqueza pero termina fallando en su afán y trae consigo inestabilidad e inflación, aún así perduran las empresas estatales que son prácticamente monopolio protegidos por el Estado. Con la llegada de López Portillo resalta su enfoque en estabilizar el país, aunque el descubrimiento de yacimientos de petróleo en el Golfo de México hace pensar al mundo y a México que este país se desarrollaría increíblemente, así el gobierno aumentó su gasto con la idea que los precios del petróleo nunca caerían.
Cuando los precios caen empieza la crisis de la deuda en México, la salida de capital, la moratoria y la consecuente nacionalización de la banca.
La primer acción del nuevo Presidente Miguel De La Madrid fue ejecutar el Programa Inmediato de Reordenación Económica (PIRE) con la idea de corregir las finanzas públicas y sentar las bases para el largo periodo de estabilización.

\section{Pacto Solidaridad Económica y Pacto para la Estabilidad y el Crecimiento Económico}
El gobierno puso mucho empeño en lograr un programa (pacto) de estabilización económica, esto se refleja en la gran cantidad de reuniones del comité junto con el Presidente para planear el programa. También destaca el trabajo de investigación realizado, donde se estudiaron más bien los casos de fracaso (Brasil y Argentina) para evitar caer en una situación similar.

\subsection{Objetivos}
El objetivo general era lograr estabilización económica para alentar así el crecimiento del sector privado; los objetivos específicos fueron generar un cambio estructural de las finanzas públicas, implementar una política monetaria restrictiva, eliminar la inercia de los salarios (generada por la indexación ex-ante y los contratos a corto plazo), coordinación de precios solo para industrias estratégicas y llevar a cabo una apertura económica.

\subsection{Ejecución}
El \textit{PSE} duró solo un año (dic. 1987 - dic. 1988) y el \textit{PECE} duró 4 años, llegando al sexenio de Salinas; se dice que el PECE fue una continuación del PSE. Algunas de las características más relevantes de los pactos fue el recorte a la carga tributaria, la privatización, disminución del gasto, rentas de empresarios y salarios reales. Como menciona Aspe, tanto gobierno, empresarios y trabajadores se comprometieron a sacrificar parte de sus ingresos económicos.

\section{Política de finanzas públicas}
\begin{itemize}%Politicas
    \item \textit{Reforma fiscal} a fondo junto con precios y tarifas púbicos a precios internacionales, el gasto del Gobierno Federal bajo control, disminuyó 8.9\% en el primer año del Pacto. Reducción de personal por medio del retiro voluntario. Reducción del gasto público, aunque aumentó el gasto social en 40\%. Hubo superávit primario durante todo el periodo del Pacto.
    \item \textit{Política de ingresos.} Los ingresos del Estado se mantuvo relativamente constante debido a la desaparición del efecto Tanzi (causado por la inflación), aumentó la recaudación fiscal debido a la disminución de la evasión y la carga tributaria.
    \item \textit{Cambio estructural.} Se privatizaron empresas que el Gobierno considero como no estratégicas, cambió la estructura de los mercados, aumentó los ingresos del Gobierno debido a las subastas y la tributación a estas empresas, se redujeron gastos ya que no era necesario mantener ciertas empresas improductivas. En suma, se privatizaron 80\% de las empresas públicas.
\end{itemize}

\section{Conclusión}
Recordando lo estudiado anteriormente, este periodo me parece un intento por salir de los problemas económicos y financieros presentados en el país a partir del desarrollo compartido; como menciona Aspe, el objetivo era conseguir estabilidad, \textbf{no crecimiento} sino estabilidad, lo cual sí se logró aunque queda a debate si esta estrategia fue efectiva, algunos argumentan que no vale la pena la estabilidad si no hay desarrollo, hay que tener en mente el contexto histórico que permitió la idea de que el mercado se desarrollaría por si mismo con liberalización \cite{AjusteMacroAspe}.