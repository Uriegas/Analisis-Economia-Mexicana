Los años de oro se caracterizaron por una participación activa del Estado con la inversión pública, bajo la idea de que el Estado era el que debía impulsar \textbf{activamente} el desarrollo, mientras que a mediados de los 80's el paradigma cambió, dejando el desarrollo a las fuerzas del mercado bajo la idea de que este asigna los recursos eficientemente y daría paso a una economía innovadora (schumpeteriana), así el Estado participaría \textbf{pasivamente}, reduciendo su tamaño a solo establecer las \textit{reglas del juego}. Aquí se presentan las las reformas (que se hizo para cambiar el paradigma) y obstáculos (los efectos negativos de este cambio) de esta transición.


\section{Reformas}
\textbf{Privatización.}
El Estado buscaba privatizar las empresas estatales argumentando que:
\begin{itemize}
    \item Daría al Estado una mejor calificación crediticia y generaría recursos por la venta de las empresas estatales (reducción del déficit fiscal).
    \item Las empresas estatales son ineficientes y el privatizarlas aumentaría su eficiencia lo que mejoraría la asignación de recursos y aumentaría el bienestar social.
\end{itemize}

\textbf{Liberalización del comercio.}
Se aplicó una política comercial basada en la teoría clásica del comercio, según la cual al dejar el mercado libre este generaría la asignación más eficiente de recursos. La intención de la política era aumentar la productividad de las empresas, lo que tendría como efecto un aumento de los salarios y un valor agregado mayor. Aunque si existieron sectores que aumentaron su productividad debido, en gran medida, a mercados industriales internos mexicanos (cadena de valor nacionales), aún así muchos otros sectores no avanzaron.

\textbf{Liberalización financiera.} 
Se formó la idea de que las reformas tendrían efecto en lo financiero, específicamente en el aumento del ahorro externo.
Con la firma del TLC se reinforzó esta idea, el TLC provocó un gran flujo de capitales hacia el país. Hay 3 factores que actuaron junto al tratado para aumentar el flujo de capital:
\begin{itemize}
    \item Liberalización de los mercados financieros internos. Abrió espacio para que inversionistas extranjeros hicieran inversiones en los nuevos mercados mexicanos.
    \item Reducción de la prima de riesgo del país. México se volvió atractivo para invertir
    \item Apreciación del peso y altas tasas de interés.
\end{itemize}

\textbf{Reformas del Estado.}
El Estado sufrió los siguientes cambios estructurales:
\begin{itemize}
    \item Se volvió más pequeño. Con la idea de que así cumpliría eficientemente sus funciones básicas, aún así su ingreso seguía dependiendo en gran medida del petróleo (un tercio) más que de los impuestos (12\% del PIB).
    \item No aumentó su eficiencia. Se esperaba lo contrario mediante la reducción del Estado (reducción de inversión pública de 10\% en 1981 a 3\% en 2001) junto con un ajuste fiscal (debido a la privatización), aún así no aumentó la eficiencia.
    \item Aumentó el gasto en política social. Debido a cambios en la distribución del ingreso.
\end{itemize}


\section{Obstáculos al crecimiento económico}
La mayoría de los obstáculos nacen de la menor intervención del Estado en la economía, la liberalización y desregularización de los mercados.

\textbf{Privatización.}
La inversión privada junto con la disminución de la inversión pública no implica un aumento en la eficiencia de los mercados, ya que hay sectores en los cuales es necesaria la inversión pública para aumentar el bienestar social.

\textbf{Liberalización del comercio.}
No hubo cambio en la asignación de recursos debido a:
\begin{itemize}
    \item Las crisis deterioró los términos de intercambio y debilitó la protección cambiaria.
    \item La productividad de las empresas no aumentó a pesar de el aumento en la especialización e intra-industria de las industrias maquiladora y automotriz; al no aumentar la productividad los salarios fueron y son bajos en estas industrias.
\end{itemize}
Un efecto de la liberalización fue la dependencia de la industria mexicana hacia las importaciones extranjeras, existiendo industrias que importan hasta 70\% de sus insumos.

\textbf{Firma del TLC}
La bonanza generada por el TLC duró pocos años dado que las inversiones eran a corto plazo, la entrada de capitales no se dirigió a la inversión sino al consumo, aumento la fragilidad financiera del país. Esto último se refleja en el déficit del 7\% del PIB de la cuenta corriente; para financiar el déficit se agotaron las reservas internacionales, dada la inestabilidad financiera fue necesario aumentar las tasas de interés de los CETES, a esto solo hace falta agregar la inestabilidad social observada en el levantamiento zapatista.
Este breve ciclo de bonanza y crisis se atribuye a que el crecimiento estaba basado en la des regularización y liberalización financiera.

\textbf{Crecimiento actual}
En el periodo 1985 - 2002 México creció solo 2.2\% anual.

Uno de los aspectos decisivos de la desaceleración de la tasa de expansión económica de México ha sido el desempeño de la inversión que tiene las siguientes características:
\begin{itemize}
    \item Racionamiento del acceso al capital y financiamiento.
    \item No se canalizaba el gasto interno hacia la inversión ( se canalizó al consumo).
    \item Eliminaban los incentivos a la inversión interna
\end{itemize}
Las anteriores características junto con el aumento de la competencia internacional y la apreciación del peso (sector externo menos atractivo) provocaron que la inversión cayera y junto a ello el crecimiento económico.


\textbf{Política Social} El aumento en  su gasto se debido a:
\begin{itemize}
    \item Aumento de la desigualdad. Esto se refleja en el índice Gini que pasó de 0.477 a 0.481 de 1994 a 2000, aunque ha aumentado desde 1984, al igual que la pobreza.
    \item Modelo de desarrollo actual. Aumentó las exportaciones (70\% después del TLC) pero no aumentó la producción del sector ejidal; la liberalización de los mercados prometía mayor eficiencia pero solo creo oligopolios por la baja competitividad.
    \item Beneficios de la globalización desiguales. Regionalmente los estados del norte se vieron más favorecidos por la exportación de mercancías a Estados Unidos.
    \item Explotación de ventajas comparativas actuales. La política de desarrollo explota las ventajas comparativas actuales más no las potenciales, a su vez el cambio a estas políticas fue brusco, dejando de lado lo que ya funcionaba.
\end{itemize}

\section{Conclusión y Pensamientos}
Los cambios de política se realizan para resolver problemas actuales, pero traen otro tipo de problemas. México se ha enfocado en el desarrollo a través de la asignación eficiente de recursos, pero este no es el problema de raíz, ejemplo de ello: la poca competitividad en libre mercado. Aunque el cambio deseado es hacia una economía estilo americana, esto requiere cambios que van más allá de las políticas \cite{ReformasHistoria}.