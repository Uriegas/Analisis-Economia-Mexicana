%Elabore un reporte en el que comente los conceptos de "populismo" y "populismo económico" y exprese su opinión. Describa las causas que originaron la devaluación de 1976. Comente el concepto de "enfermedad holandesa" que se presenta en este capítulo y explique su relación con el incremento de los precios del barril de petróleo durante los últimos años de la década de los setenta.

Durante el sexenio de Echeverria y López Portillo surgió lo que se conoce como el desarrollo compartido, impulsado primeramente por Echeverria causando problemas económicos que fueron en posterior resueltos a medias por López Portillo que dejándose llevar por la euforia del descubrimiento de yacimientos petroleros y sus altos precios dañó la economía nacional aumentando el gasto por medio del crédito, el cual no pudo pagar al difuminarse la euforia.
En este reporte se presenta de manera concisa los puntos claves que explican cómo es que el país paso de una época de bonanza con el desarrollo estabilizador a un periodo de constante crisis y gran gasto financiado con deuda. Finalmente el autor concluye con una reflexión sobre las intenciones de los presidentes y sus lamentables efectos.

\section{Fin del desarrollo estabilizador}
Se considera que el populismo es la idea de mejorar el bienestar de un grupo de la población a costa de otro, estas políticas de redistribución suelen realizarse con el fin de tener un apoyo político, aunque a corto plazo benefician a un sector de la población, en e largo plazo dañan la economía de todo un país. Por otra parte, se considera populismo económico a la política expansionista implementada por un gobierno que termina generando inflación y/o devaluación de la moneda, lo cual se ve reinforzado con la aplicación de políticas keynesianas mal interpretadas, con las cuales se espera que al aumentar el gasto se llegará a una situación de pleno empleo.
Con la llegada de Luis Echeverria a la presidencia, el enfoque económico cambió de un desarrollo estabilizador a uno compartido, la idea era distribuir los ingresos (que ya eran altos y constantes en el país) de una manera más equitativa. Se buscaba beneficiar a los más necesitados a costa de aquellos que se habían enriquecido anteriormente (aunque no se dijo abiertamente fue lo que terminó pasando al corto plazo).
La estrategia consistió en aumentar el gasto público atendiendo problemas sociales y generando empleo, esto se financió con endeudamiento externo y expansión monetaria ya que se deterioró la relación entre el sector público y privado, a raíz del discurso político.

\section{Crecimiento por gasto público}
El alto gasto público no fue intencional en un principio (en el presupuesto de 1971 se redujo el gasto), sino una medida contra cíclica para revertir la caída del crecimiento económico, aún así siempre existió la tentación de imprimir más dinero sobre todo manteniendo ideas como la anti-exportación. Eventualmente el gasto público fue aumentando, algunas veces con endeudamiento y otras con la monitorización del mismo por parte del banco central, no solo se aumentó el gasto en 21.2\% en el presupuesto, sino que fue un gasto ineficiente, las decisiones económicas se trasladaron a los pinos y el presidente era el que tenia la última palabra sobre el rumbo de la economía, la economía si creció aunque muchas veces con altos números de inflación.
El gobierno gastaba más de lo que ingresaba (problema que ya se venía presentando poco antes del fin del desarrollo estabilizador) de forma que se buscó introducir de nuevo la reforma fiscal de 1964, aunque sin éxito, finalmente solo se aumentó un poco un impuesto mercantil. El deterioro del sector privado generó una fuga de capitales; además la moneda se sobre-valuó por la política expansionista, generando un aumento de importaciones y una reducción de exportaciones.

\section{Auge petrolero y colapso del 82}
Después de la transición presidencial a López Portillo su plan fue primero acabar con la recesión, estabilizar la economía y finalmente recuperar el crecimiento económico. Estos planes que comprendían todo su sexenio no duraron mucho, ya que se descubrieron grandes yacimientos petroleros en México, lo cual junto con los altos precios del mismo en el mercado internacional generó una euforia internacional sobre el futuro del país, llegando a hablarse de una administración de la abundancia.
El país se recuperó rápidamente de la crisis, una vez más se aumentó el gasto, con la diferencia de que ahora se recuperó el efecto \textit{crowding in} que se tenia antes del periodo de Echeverria, se reabrieron las líneas de crédito exterior, también se recupero el sistema financiero y la fuga de capitales, todo esto provocó que el gobierno aumentara el gasto público el cual se dirigió a expandir a PEMEX, la mayor parte de los ingresos de esta paraestatal se dirigieron a su expansión. Quedo en entredicho la eficiencia del alto de gasto del periodo; aún con todo ello se deterioró la balanza de pagos lo cual daño eventualmente a la economía, esto debido a las altas reservas de dolares generadas por el petróleo, las cuales generaron una sobre-valuación del peso que hizo más baratas (competitivas) las importaciones con respecto a los productos nacionales, y a la vez encareció los bienes de exportación nacionales (menos competitivos internacionalmente); esta situación daño la industria local convirtiendo a la economía en una economía petrolificada, este fenómeno se conoce como la \textit{enfermedad holandesa}.

La euforia petrolera en México no podría durar para siempre, y es que el modelo de crecimiento de aquel tiempo fue derrochador, un alto gasto público con la premisa de que el precio del petróleo nunca caería. Aunque en realidad si cayó, si bien no fue una caída repentina, si fue lo suficientemente fuerte como para mover los mercados y al gobierno nacional; se puso en entredicho la capacidad de México para continuar su modelo de desarrollo y por lo tanto de hacer frente a sus obligaciones. El déficit era tal que ya representaba el 14\% del PIB, las deudas contraídas anteriormente debían continuarse pagando, para hacer frente a este desequilibrio primeramente se dejó actuar a la inflación y a la vez el tipo de cambio se redujo de 26 a 47 pesos, el gobierno no podía seguir accediendo a crédito externo, por lo que llegó a un punto en el que suspendió los pagos de deudas (moratoria) y terminó por nacionalizar la banca. Esta fue la última política del Presidente López Portillo; consecuencia de un gasto desmesurado financiado con deuda con la idea de que México debía administrar la abundancia, la cual nunca llegó.

\section{Conclusión}
Pienso que existen diferentes intenciones para que un político llegue a ser populista, por una parte la intención de mantener un poder político pudiendo llegarse a una dictadura, por otra parte una verdadera intención de mejorar la situación de los más necesitados y del pueblo; aunque es difícil conocer las verdaderas intenciones pienso que en el caso de Luis Echeverria tenía verdaderas intenciones de mejorar la situación del país después de observar por años un alto crecimiento económico que solo beneficiaba a unos pocos, aún así no basta con las intenciones y las políticas llevadas no solo no distribuyeron la riqueza sino que la detuvieron; aunque trato de mantener mis análisis económicos fuera de la política actual, debo admitir que en este caso el alto gasto que se esta haciendo en programas sociales pienso que es una política populista de distribución de los ingresos, de la cual desconozco las intenciones pero seguramente garantizarán un apoyo político; por otra parte el periodo de López Portillo lo veo basado en una promesa falsa de prosperidad que fue alentada por los altos precios del petróleo; resulta fácil juzgar ahora pero en aquel tiempo verdaderamente se creyó que México saldría de la pobreza e imagino la responsabilidad que sentía que si no actuaba correctamente dejaría caer esta oportunidad en manos extranjeras o de unos pocos \cite{PopulismoMexicano}.