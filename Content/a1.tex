%%Cada sección es una pregunta%%
\textbf{Reporte.}

El brote de Covid 19 en China se transmitió rápidamente alrededor del mundo entero, esto tuvo y esta teniendo efectos en la economía global, los precios de los productos básicos, los mercados financieros y ha cambiado los hábitos de consumo y las medidas de la política económica. Aquí se presenta un recuento de los efectos del Covid en todas estas áreas.

\textbf{Producción}

El 90\% de los países del mundo presentaron recesión en el año 2020, los principales países afectados fueron las economías desarrolladas como Estados Unidos y la Zona del Euro, con proyecciones de caída del PIB del -6.5\% y -8.7\% respectivamente.
Llama la atención el aumento de producción industrial y las ventas minoristas, 5.4\% y 18\% respectivamente, esto último relacionado con el desempleo, el cual tuvo una caída del 11.1\% en junio, este desempleo obligó a muchos a dedicarse al menudo.

\textbf{Comercio Mundial}

El comercio mundial disminuyo poco (0.4\%) aunque la OMC proyecta una abrupta caída de 13 a 32\% para el 2020.
El sector más golpeado fue el sector servicios, principalmente el turismo, con una caída del 56\% y el transporte (global y local), esto debido al confinamiento que impuso restricciones sobre el movimiento de las personas.
Se dañaron los términos de intercambio para los productos básicos, por lo que países exportadores de estos productos tendrán menor poder de compra con el mismo nivel de ventas.

\textbf{Precios de Productos Básicos}

Debido a la baja movilidad el transporte dejó de usarse durante un buen tiempo globalmente, lo cual ocasionó una caída en el precio del petróleo, con los históricos precios negativos del petróleo WTI, debido a la falta de espacio para almacenar petróleo ya extraído; se proyecta que el petróleo disminuya un 36\% en el 2020.
Por otro lado, los precios de los metales preciosos aumentaron, esto debido a que son usados como depósitos de valor en tiempos de crisis, el oro por su parte incremento un 28\% con respecto al 2019.
Aunque los precios de los alimentos ha disminuido (1.5\%), no ha sido tan representativo como la caída de la energía, aunque en general el sector primario disminuyó sus precios.

\textbf{Política Económica}

La participación de los bancos centrales en esta crisis es comparable a la participación en la Gran Depresión, en general, se presentó una gran flexibilidad por parte de los bancos centrales, quienes buscaron de llenar de liquidez los mercados financieros.
Esta liquidez se logró a través de la reducción de las tasas de intereses y la expansión de la base monetaria, esta última mediante la acuerdos de recompra (repo), donde se cambiaron activos (títulos de deuda, etcétera) por dólares y líneas de crédito recíproco (swaps), donde la Reserva Federal creó dólares para luego intercambiarlos con bancos centrales alrededor del mundo.
Por otra parte la expansión de la oferta monetaria trajo consigo programas de recuperación económica por parte del gobierno y organismos internacionales, así como estímulos fiscales para mantener a los negocios afectados por la pandemia. Estos programas se lograron con el aumento de la deuda de los Estados, lo cual puede agravar la situación posterior de países que antes de la pandemia ya tenían altos niveles de deuda.

\textbf{Mercados financieros}

En los mercados financieros ha aumentado la inestabilidad y volatilidad de los mercados, esto a pesar de la confianza que intentan mantener los bancos centrales con la expansión monetaria.
Aún así persiste mucha incertidumbre sobre lo que pasará en el futuro, lo cual general la volatilidad e inestabilidad, que se propaga a todo el sistema financiero.

La incertidumbre se evidencia en la caída que han tenido los mercados bursátiles, pero a su vez un aumento en los últimos días. Por su parte el dólar se ha apreciado con el comienzo de la cuarentena debido a la necesidad de dólares en el sistema, lo cual se suplió con los repo y swaps de la Reserva Federal, por ello mismo se observó una depreciación del dólar. Aún así, tanto los mercados como el dólar se han mantenido volátiles enfrente de un escenario de incertidumbre.

La diferencia de esta crisis radica en que fue una crisis en el sector real de la economía que después se transmitió al sector financiero, caso muy diferente a la crisis del 2008, en la cual los problemas del sector financiero se transmitieron al sector real \cite{CEPALCovid}.