%Elabore un reporte en el que describa el comportamiento de los principales indicadores de la actividad económica durante los años 1963-1971. Mencione lo que el autor considera fuentes de la debilidad económica, haga referencia a los factores que inciden en la pérdida de la competitividad del sector industrial y mencione las características del sector agropecuario durante ese periodo.

Durante la década de 1963 - 1971 se vivió en México lo que se conoce como el desarrollo estabilizador, periodo marcado por un alto crecimiento económico y que inclusive es tomado como modelo de desarrollo por el presidente actual \cite{AmloDesarrolloEstab}, aunque es verdad que fue un periodo de bonanza este presentó problemas que bloquearon el futuro desarrollo. Este reporte se divide en 3 secciones: la primera presenta la situación de desarrollo, la segunda los problemas presentados, la tercera una conclusión personal.

\section{Panorama General}
La economía creció 7.1\% anualmente, mientras que la inflación lo hizo al 2.8\%, además el PIB per capita creció 13.6\%, lo que significó un aumento del poder adquisitivo del ciudadano promedio, de hecho el salario real creció, así como la relación capital/trabajo (7\%). Dentro de las características importantes de la composición de la economía están que hubo un flujo poblacional importante hacia las zonas urbanas, así como los sectores agrícola y minero perdieron su participación en la economía, siendo remplazados por sectores más industriales como el eléctrico que llegó a crecer 14.2\% anualmente; también se observó una relación importante entre la inversión pública y privada, donde la primera promovía la segunda, normalmente debido a que la primera era inversión en infraestructura necesaria para el desarrollo de la segunda. Otra parte importante fue el sector externo que también perdió representación en el PIB pasando de 25.3\% a 20.3\%, esto por el enfoque anti-exportaciones del periodo y un modelo de sustitución de importaciones con el fin de desarrollar la industria interna. Por último, hubo un importante desarrollo del sector financiero que permitió la auto financiación del desarrollo, específicamente con el aumento del ahorro interno y la emisión de bonos gubernamentales introducidos exitosamente al sistema financiero, claro esta que existieron sus limitantes en este sector.

\section{Fuentes de debilidad económica}
Aunque el periodo fue de crecimiento y dejo lecciones para implementar una política económica de desarrollo nacional, este también dejo lecciones sobre lo importante que es no descuidar ciertos sectores, posponer decisiones y lo difícil que resulta la toma de decisiones en un ambiente político que muchas veces olvidamos al realizar un análisis económico de un periodo que ya ha pasado.
Los siguientes puntos fueron las principales fuentes del debilitamiento de la economía y la no sostenibilidad del modelo (el cual a largo plazo siempre debe estar abierto al cambio en mi opinión, aunque siempre existe fricción al cambio):
\begin{itemize}
    \item Baja competitividad. Parte del modelo de desarrollo se basaba en el proteccionismo de la industria nacional, específicamente mediante estímulos fiscales como subsidios en el coste de la electricidad y el transporte en ferrocarriles. El problema de este modelo fueron los siguientes: en las primeras etapas de desarrollo de una industria hace sentido protegerla de la competencia externa, pero una vez que ha crecido debería abrirse la competencia para disminuir costos e incentivar la competencia, lo cual no se hizo en el caso nacional; esta decisión no se tomó en parte por un entorno político complicado y el miedo a que al liberar el mercado las recién maduras industrias desaparecieran por la competencia de industrias ya bien cimentadas en el extranjero dejando al país no solo con una industria menor, sino con un mayor desempleo; por otra parte, el país tuvo malas históricamente tuvo malas experiencias con el tipo de cambio y las exportaciones agrícolas. Por último, se inició un proceso de \textit{mexicanización} de la economía, donde el gobierno tuvo un sesgo aún mas fuertemente anti-exportador con un enfoque en el mercado interno, de tal forma que muchas industrias llegaron a ser auto sustentables, este sesgo generó la poca competencia, además de un déficit fiscal debido a que no se podía financiar las importaciones de bienes manufacturados.
    \item Deterioro del sector agrícola. En décadas pasadas el desarrollo fue impulsado por los sectores agrícola y manufacturero, la importancia de la agricultura fue que otorgaba productos esenciales a bajos precios y que además exportaba, lo que generaba los ingresos para la formación de capital para la industria manufacturera. Este sector llegó a contribuir con el 15.4\% del PIB y el 49.9\% de las exportaciones, mientras que en los 60's representó solo el 7\% del PIB con un crecimiento del 2.3\% anual. Este cambio tan repentino se debió a 3 factores principales: primero, la disminución de los precios internacionales del algodón por parte de la liberalización de este mercado en Estados Unidos, siendo este el principal bien exportador de México; segundo, hubo una recomposición de la inversión pública la cual dejó de lado la inversión en el sector minero para realizar importantes inversiones en infraestructura y en el sector manufacturero; tercero, en un principio hubo un aumento de la población en las poblaciones rurales, lo cual disminuyo el área de cosechas, una vez el campo se deterioró la población empezó a emigrar a las ciudades aumentando así la desigualdad entre el campo y la ciudad.
    \item Dependencia financiera externa. A pesar de los desarrollos en los importantes desarrollos en el sistema financiero y su \textit{mexicanización} persistió una dependencia al crédito externo, crédito debido a que el Estado consideraba mejor el crédito externo que la inversión extranjera, inclusive se aumentó más el proteccionismo ya no solo a los bienes extranjeros sino al capital extranjero, el gobierno otorgaba apoyo a aquellas empresas en las que la mayoría del capital accionario fuera nacional. Para poder financiar la falta de capital en la inversión pública como privada se buscó aplicar las siguientes medidas: primero, aumento de los impuestos, es decir, reforma fiscal, esto debido a deficiencias en la tributación mexicana finalmente no se aplicó una reforma fiscal por el entorno político; segundo, se financió el déficit con el apoyo del banco central, es decir, se monetarizó el déficit, lo que produjo inflación; tercero, el gobierno aumentó su ingreso mediante el aumento de los precios de bienes y servicios de públicos y ofrecidos por paraestatales.
\end{itemize}

\section{Conclusión}
Me parece que cada modelo tiene sus pros y contras, aunque fue un periodo de alto crecimiento los problemas que permitió aún son latentes en la sociedad mexicana. Pienso que parte del desarrollo se debe en parte a la urbanización que trae consigo una necesidad en la inversión de infraestructura. En lo personal me llama la atención como desde el periodo post-guerra se observa una inversión pública que genera un aumento de la inversión privada, algo similar podría ocurrir con la inversión en infraestructura en telecomunicaciones (5G) y los posibles casos de negocio que puede generar en la industria y la economía digital \cite{DebilidadEstructural}.