%Elabore un reporte en el que comente la relación que el Estado establece con la economía, las políticas de impulso económico, las características de la estructura del producto, así como las relaciones que se establecen entre el Estado y las clases sociales para crear un nuevo estilo de crecimiento económico. Hasta la 265

El periodo de 1940-1954 se caracterizó por un alto crecimiento económico impulsado por el Estado, creando así las condiciones para el desarrollo. Para lograrlo se requirió de la colaboración entre Estado, sociedad y sector privado para impulsar un proyecto de Nación. Aquí se analizan las distintas maneras en que el Estado impulsó el desarrollo económico con un enfoque en el proyecto de nación que es inclusivo. Se concluye con una opinión personal sobre este desarrollo y las lecciones aprendidas.

\section{Estado como catalizador del desarrollo}
La causa del crecimiento económico de 6\% anual fue el gasto público mediante la aplicación de política expansionista y substitución de importaciones. Aquí se mencionan puntos claves para el financiamiento del gasto público:
\begin{itemize}
    \item Reducción de la carga fiscal en el sector industrial. Los ingresos por carga fiscal en este sector se redujeron de 21\% a 10\% de 1940 a 1954.
    \item Sistema financiero. El gobierno recurría en primera instancia al Banco Central, con el tiempo se formó un mercado de bonos gubernamentales que le permitieron al gobierno accesar a más recursos, además de que accedía a otras instituciones, entre las que resalta Nacional Financiera que le permitió incurrir en déficit.
\end{itemize}

\section{Características del proceso de desarrollo}
\begin{itemize}
    \item Ampliación mercado interno. Dada la reducción del mercado internacional por la guerra, la política de substitución de importaciones y la migración hacia las zonas urbanas provocaron un incremento en la demanda interna y la producción nacional.
    \item Modernización ligada al avance industrial. Por la aparición de un nuevo tipo de empresariado y su colaboración con el Estado.
\end{itemize}

\subsection{Sector agropecuario}
Se consolidaron las relaciones capitalistas, se fortaleció la propiedad privada, creció la producción de productos de exportación, principalmente en el Norte, aún con un aumento de la demanda interna; finalmente el sector se volvió atractivo por su rentabilidad.

\subsection{Demanda Interna}
La situación internacional junto con las políticas implementadas aumentó la demanda de bienes de consumo final, dicha demanda fue satisfecha por empresas nacionales (dada la baja oferta internacional) lo cual impulsó el desarrollo de la industria mexicana, principalmente las industrias de construcción, transformación, electricidad y combustibles.

\subsection{Estrategia ante restablecimiento de mercados globales}
El crecimiento de los mercados internacionales después de la guerra representó una amenaza al crecimiento económico mexicano, esto se refleja en el déficit generado en la balanza comercial. Para hacer frente a esto el Estado implementó proteccionismo y más gasto público, con un enfoque en impulsar la industria nacional mediante substitución de importaciones, lo cual cambió la estructura de la industria en el futuro. Aún así se generaron problemas con el aumento de la migración hacia las urbanizaciones creando empleos informales, generando dificultades con el sostenimiento del empleo y el salario.

\section{Pacto Social y Proyecto de Nación}
Eran necesarios cambios para pasar de una sociedad revolucionaria a una regida por instituciones solidas, el Estado tomó un rol activo creando las condiciones para el desarrollo. El Estado modificó sus relaciones con los siguiente grupos:
\subsection{Empresariado}
Promoviendo activamente la inversión privada y garantizando sus ganancias.
Anteriormente existían pocos grupos de empresarios los cuales tenían un alto poder político y económico, y no veían con buenos ojos la involucración del Estado en la industria, aún así en los últimos años de este status quo existieron 2 tipos de empresarios:
\begin{itemize}
    \item \textbf{Tradicional.} Grupo de financieros y comerciantes que se dedicaban a importar productos y desarrollar industrias tradicionales, tenían la ventaja de contar con el apoyo del capital extranjero. Tenían una visión internacional del desarrollo, pesaban en aumentar las exportaciones de aquellos bienes competitivos internacionalmente e importar aquellos con los que no se contaba o no se podía competir.
    \item \textbf{Nuevo Grupo.} Comprendía a los dueños de pequeñas fabricas que satisfacían la demanda interna y que no tenían buenas relaciones con los bancos. Pensaban que el desarrollo se lograría por medio de la industrialización del país, tenían la visión de una involucración de los negocios al Estado y no al revés.
\end{itemize}
El Estado aplicó la técnica \textit{divide y vencerás}, mediante leyes dividió los grupos dándole a los últimos representación política en la Cámara Nacional de la Industria de Transformación. El gobierno no dejó fuera al grupo tradicional aunque si existieron fricciones con ellos.

\subsection{Obreros, agrarios y otros grupos sociales}
Después de mejorar las relaciones con el empresariado se hizo lo mismo con los obreros con el fin de mantener el orden, esto se logró mediante instituciones como: Secretaria del Trabajo, Comisión Tripartita y las reformas a la Ley del Trabajo.
Estos logros dieron control sobre los sindicatos para mantener las ideologías del movimiento obrero bajo control y creó una paradoja en los sindicatos: para que se cumpliesen sus exigencias debía mantener su relación con el Estado, pero a la vez esta relación reducía el cumplimiento de sus exigencias. Eventualmente la izquierda radical perdió su influencia.
El Estado también mantuvo bajo control el movimiento agrario con la ayuda del desarrollo de los pequeños productores y la buena imagen que trajo el reparto agrario.
El resto de los grupos sociales se organizaron en la CENOP, manteniendo el orden en la educación y la salud; dentro de la CENOP se formaron políticos que en el futuro influirían en el rumbo del país.

\section{Poder Ejecutivo}
Los cambios más importantes en la estructura política fue la instauración del orden y las instituciones, desapareció el poder de caudillos y militares.
En esta transición existió inestabilidad política relacionada con ideologías heredadas de la revolución dentro de grupos como el ejercito y la cámara de diputados; el presidente adquirió una posición de arbitro siendo pieza clave para decidir el rumbo del desarrollo \cite{EstadoyDesarrolloUNAM}.

\section{Conclusiones y lecciones aprendidas}
Este periodo se conoce como el milagro mexicano y se observa que fue impulsado por el Estado, llama la atención lo incluyente y estratégico que fue, impulsó la educación, salud, industrialización y el institucionalismo. Me parece que para el desarrollo debe existir un sentido común de lo que queremos lograr como país, y este debe sentirse entre los 3 ejes: gobierno, empresariado y trabajadores. El entorno internacional favoreció al país económicamente, pero creo que lo hizo más las lecciones de unidad, solidaridad y lucha por lo que se cree, las cuales se aprendieron en esa época.
