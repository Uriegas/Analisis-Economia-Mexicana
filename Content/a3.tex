\textbf{Reporte.}

La Encuesta Nacional de los Ingresos y Gastos de los Hogares (ENIGH) recopila los datos económicos de los hogares mexicanos tomando en cuenta las características de estos hogares, como tipo de vivienda, decil de ingreso, número de habitantes, etcétera. Aunque la edición 2018 es la encuesta que más datos ha recopilado al día de hoy, solamente se encuestaron a 87,826 hogares, considerando que la población nacional es de 125,091,790 habitantes.

En promedio hay 3.6 personas por hogar, donde el 2.38 perciben ingresos, pero solamente el 1.7 trabajan, la diferencia entre percepción y ocupación económica se puede deber a las becas de los estudiantes y a las remesas de los adultos mayores y mujeres, entre otros.

\textit{Ingresos.}

La principal fuente de ingreso de los mexicanos es el trabajo, representando un 67.3\% del ingreso, seguido de las transferencias (5.4\%), alquiler (11.4\%) y rentas (5.9\%).

Es remarcable el ligero cambio en la distribución del ingreso, donde de 2016 a 2018 cayó el ingreso del décimo decil en un 7.8\%, mientras que los deciles del II al IV presentaron incrementos del 5.8\%.

Por consiguiente el índice Gini disminuyó de 0.475 a 0.426 de 2016 a 2018, sin considerar las transferencias el coeficiente a 2018 sería de 0.448; aquí se observa una vez más el efecto positivo que tienen las transferencias (remesas) a la distribución igualitaria del ingreso, dado que estas van principalmente a hogares que se encuentran en los deciles bajos o medios.

Por otro lado, las áreas urbanas tienen un mayor ingreso que las áreas rurales, específicamente 1.8 veces el de las áreas rurales. A esto se debe agregar las desigualdades en el ingreso por grupos de edad, genero y grado escolar, donde las personas entre 50 y 59 años son las que mayores ingresos tienen, mientras que las mujeres en general presentan un menor ingreso que los hombres, y entre un mayor nivel de estudios se presenta un mayor nivel de ingresos.

\textit{Gastos.}

Dentro de los gastos se observa que estos se encuentran más \textit{diversificados} que los ingresos, estos últimos provenían principalmente del trabajo, mientras que los gastos se distribuyen entre alimentos (35.3\%), transporte (20\%), vivienda (12.1\%) y educación (9.5\%) principalmente.

Por su parte, los alimentos (que es el mayor gasto), se realizan principalmente en alimentos en el hogar, lo cual dista de la situación de otros países, en los cuales la alimentación se realiza se realiza fuera del hogar. Esto no es precisamente malo, aunque si es preocupante los tipos de alimentos que se consumen, dado que solamente el 11.7\% de estos son frutas y verduras.

En general, se observa muy poco gasto en cuidados de la salud, representando solamente un 2.4\% para los hogares urbanos y 3.7\% para los hogares rurales. Por otra parte, el gasto en la vivienda es mayor en los hogares urbanos que en los rurales.

Dentro de las variaciones más importantes del gasto en el periodo 2016 a 2018 se encuentra el gasto en combustibles para vehículos, el cual tuvo un aumento del 13.7\%, mientras que la mayor caída en el gasto fue para el esparcimiento, con un crecimiento de -8.4\%.

\textit{Distribución por entidad federativa}

En cuanto a la distribución regional del ingreso y el gasto, los estados con mayor ingreso y gastos son Ciudad de México y Nuevo León, mientras que el estado con menor ingreso y gasto es Chiapas, se observa cierta correlación entre la urbanización y el nivel de ingresos, donde los estados más urbanizados también tienen un mayor ingreso y a su vez un mayor gasto.

En conclusión, el ENIGH muestra los datos "semi-crudos" de la situación en la que viven los hogares mexicanos; con la crisis actual podemos suponer que existen algunos cambios que repercutirán en el ingreso y gastos de los mexicanos; aquí resaltan 3 aspectos importantes desde mi punto de vista:
\begin{enumerate}
    \item \textbf{Remesas.} Como se mencionó, parte de la "buena" distribución del ingreso se debe a las transferencias, que aunque no se menciona, es muy probablemente las remesas de mexicanos trabajando en el extranjero a sus familias en el territorio nacional; con la crisis de hoy día, parte de los mexicanos en el extranjero observaron una reducción de sus ingresos o bien la terminación de los mismos, lo cual se verá reflejado en la siguiente edición del ENIGH.
    \item \textbf{Distribución de ingreso por edades.} Es una situación que se debe de analizar en el largo plazo, dado que los adultos entre 50 y 59 años son los que más ingresos reciben, mientras que los jóvenes de 12 a 19 años son los que menos ingresan, esto no es precisamente malo; pero si presentaría un problema de financiamiento de los adultos mayores en un futuro en el que México empiece a envejecer, este problema ha sido analizado en Estados Unidos, y se muestra como un reto de carácter político principalmente\footnote{Ver "No Free Lunches" - Stan Druckenmiller: \href{https://youtu.be/DXAEw8psMuQ}{https://youtu.be/DXAEw8psMuQ}}.
    \item \textbf{Gasto en Salud.} Existía un bajo nivel de gasto para el rubro salud por parte de los mexicanos, lo cual pienso que cambiará debido a la mayor preocupación por la salud que ha creado la pandemia.
\end{enumerate}
\cite{ENIGH2018}