La política de desarrollo social actual cambió en el 2019 con la entrada de la nueva administración del Gobierno Federal, aquí se presenta la situación de la población mexicana en cuanto a desarrollo social considerando la pandemia actual; a su vez se presentan los rasgos de la política actual y se evalúa su efectividad0.

\textbf{Evolución de la pobreza multidimensional en el periodo 2008-2020.}

La pobreza en México sigue siendo alta (41.9\%), aunque ha disminuido un poco, pero esto se debe a una transferencia del tipo de pobreza de extrema a moderada, se observa que la incidencia en la pobreza es mayor en zonas rurales que en urbanas. Actualmente existe una disminución del empleo formal, lo cual trae problemas sociales, esto ya que es más sencillo entregar apoyos canalizados por el nivel salarial siempre y cuando sea empleo formal.
En cuanto al ingreso, este se ha visto muy afectado desde la crisis del 2008, solamente en 2020 estuvo apunto de regresar a tal poder adquisitivo.
Con la crisis actual la pobreza laboral aumentó a poco más de la mitad de la población; también aumentó la informalidad y el desempleo, se presentan retos para combatir la pobreza en México, dado que ya no es solamente pobreza extrema o marginación, sino pobreza moderada relacionada con la perdida de empleos.

\textbf{Condiciones socio económicas de los grupos vulnerables del país.}

A continuación se presentan los grupos sociales que han sido históricamente vulnerables:
\textbf{Mujeres.} Históricamente discriminada, tiene un menor acceso al empleo formal, por lo que no tienen acceso a un sistema de protección social; a su vez la brecha salarial aumenta más en la situación de pobreza extrema, las mujeres emplean más horas en los quehaceres de la casa y cuidado de otras personas, trabajo no remunerado. 14\% no tiene acceso a servicios de salud.

\textbf{Población indígena.} Alrededor del 70\% de la población indígena esta en pobreza, a su vez estos carecen de seguridad social a un nivel mayor que el de la población en general. También hay rezago educativo, lo que limita las posibilidades de esta población. El aumento de la población indígena en el empleo es engañoso, dado que este empleo es principalmente con condiciones precarias.

\textbf{Personas con discapacidades.} 4 de cada 10 personas con discapacidades son parte de la población económicamente activa (PEA), mientras que de la población normal 7 de cada 10 son parte de la PEA. Esto muestra la disparidad en el acceso al empleo de las personas con discapacidades. Solamente 20.2\% tiene acceso a la seguridad social.

\textbf{Niñas, niños y adolescentes (NNA).} 49.6\% de los NNA se encuentran en pobreza, mientras que un quinto de los NNA de México no tienen acceso a la canasta básica; esto causa problemas de nutrición tanto para aquellos sin acceso a la canasta básico (desnutrición 13.2\%) como a aquellos con acceso (obesidad 14\%). Por la crisis pronostica que los embarazos no deseados aumenten considerando el poco acceso a anticonceptivos, se pronostican 191,948 embarazados no deseados.

\textbf{Población joven.} 40\% esta en situación de pobreza y 67.2\% no tiene seguridad social (siendo el más alto de todos los grupos). Por la crisis puede presentarse un aumento en el desempleo juvenil, creando así más obstáculos a la entrada al mercado laboral para los jóvenes; esto puede repercutir en el derecho al acceso a un trabajo digno.

\textbf{Personas mayores.} Sufren discriminación por estereotipos en cuanto al trabajo principalmente. Existe mayor pobreza en la población mayor de 65 años que no tiene acceso a la seguridad social, tomando esto en cuenta 17.7\% no tiene acceso a seguridad social y 76.1\% están empleadas informalmente.

\textbf{Enfoque territorial de la política social.}

Para canalizar los recursos destinados a la política social en México se emplean diversos indicadores, uno de ellos es el ZAP (Zonas de atención prioritaria), el cual tiene un enfoque territorial a nivel municipio basado en 3 variables:
\begin{itemize}
    \item Porcentaje de población indígena. Existen 623 municipios con alta población indígena, principalmente en el Sur y Sureste del país. Las poblaciones indígenas presentan en general mayores niveles de pobreza multidimensional y marginación históricamente.
    \item Nivel de marginación. Existen 1100 municipios en marginación, la mayoría en Chiapas, Guerrero y Oaxaca, donde gran parte de ellos son heterogéneos, es decir, tienen diferentes tipos de marginaciones. La desventaja de este indicador es que no considera cambios en el tiempo, por ejemplo en la crisis actual el país ha empeorado, aún así los municipios marginados siguen siendo los mismos, aunque existan otros que estén aumentando su nivel de marginación, esto genera un sesgo en la implementación de la política social.
    \item Índice de violencia. Este índice considera la incidencia de 4 delitos: homicidio, extorsión, secuestro y robo de vehículo, donde la mayoría de los delitos considerados en este índice son de robo de vehículo (82.9\%). El índice da el mismo peso a los 4 delitos, por lo que un municipio con solo altos homicidios y otro con solo alto robo de autos tendrán el mismo valor en el índice; este problema se vuelve más grande al ver que el índice solo considera las denuncias de estos delitos y no los que han sucedido en realidad, tomando en cuenta también que el nivel de denuncia de delitos en México esta entre 4.1 y 18.1\% (bajo); este es un problema global dado que es difícil medir el nivel de violencia de una localidad aún así hay cambios metodológicos que se podrían realizar.
\end{itemize}
Como se observa los municipios beneficiados con el enfoque territorial son principalmente las comunidades indígenas del sur y los municipios con mayores niveles de robo de autos.

\textbf{Características de política social para el periodo 2019-2024.}

La política social del actual gobierno federal tiene como objetivo generar un bienestar para toda la población desde la perspectiva de derechos humanos. La política cambia de paradigma de programas y acciones separadas a un enfoque integral de bienestar universal. Para llegar a ello se busca intervenir a las regiones históricamente marginadas.

El programa integral de bienestar se enfoca en 2 aspectos: atención a todas las etapas del ciclo de vida, autosuficiencia alimentaria. La primera se subdivide en la prevención, mitigación y atención de riesgos a la salud y riesgos sociales (ingreso y des. humano), en la practica la atención se enfoca a mitigar problemas sociales de ingreso primordialmente. El segundo esta relacionado con la agricultura como medio para combatir el hambre y la pobreza mediante la autosuficiencia, aunque en la practica no existe mucha coordinación.
Al evaluar la coordinación entre la CPGD y SEBIEN se encontró una estrategia conjunta en la operación coordinación en la ejecución, en general se evalúa como buena la coordinación en los pasos de la administración, aunque en la planeación no existe mucha coordinación por parte de la CPGD y se puede mejorar la burocracia. En cuanto a la crisis actual los programas sufrieron modificaciones para atender las necesidades actuales (21 de 53), se aumentó la atención médica a los grupos vulnerables \cite{CONEVALDesarrolloSocial}.