%Elabore un reporte en el que comente la política económica instrumentada durante el Pacto para la Estabilidad y el Crecimiento Económico (PECE): renegociación de la deuda externa, política monetaria, políticas de concertación de precios, comercial y cambiaria. Finalmente, señale el comportamiento de la inflación, empleo y producción durante el Pacto.
El Pacto para la Estabilidad y Crecimiento Económico tuvo un enfoque en estabilización y cambio estructural de la economía, ello para resolver los problemas heredados y sentar una base para el crecimiento económico futuro a la vez, hoy día muchas de las características de la economía mexicana surgen de esta época de reestructuración.
Aquí se explica cómo actuó el programa en 5 rubros clave y sus efectos. Finalmente, se presentan las ideas económicas de Aspe y una conclusión sobre política económica por parte del autor.

\section{Renegociación de la deuda}
En un principio los acreedores (ej. FMI y Banco Mundial) se negaban a renegociar la deuda durante el último periodo de Miguel de la Madrid debido a que no confiaban en la eficacia y continuidad del trabajo que se estaba realizando (PECE), esperaron a la transición política, y con la entrada de Salinas se renegoció la deuda con una disminución equivalente al descuento en el mercado secundario. La renegociación permitió revertir las transferencias netas de positivas (dinero sale del país) a negativas (dinero entra), lo que se refleja en le aumento de las exportaciones y la inversión externa.

\section{Política monetaria}
Las tasas de interés reales subieron, lo cual se puede atribuir a dos causas muy diferentes:
\begin{itemize}
    \item \textbf{Expectativa.} La caída de la inflación esperada y la reducción de la inflación generaron exceso de demanda de dinero y el aumento de las tasas de interés reales.
    \item \textbf{Riesgo.} Aumento debido al riesgo de invertir en una economía en transición.
\end{itemize}

Cada explicación del aumento de las tasas conlleva políticas divergentes, la primera conlleva monetizar, mientras que la segunda impedía (no era sabio) el aumento de la oferta monetaria. 
El desafío de la política monetaria era resolver 2 problemas: estrangulación crediticia y desplome del tipo de cambio.
Ante ello el banco central decidió reducir gradualmente los flujos de crédito al gobierno (dejar de financiar con expansión monetaria al gobierno), lo que tuvo como efecto que el gobierno ya no acude al banco central por crédito, sino a los mercados financieros hasta hoy día.

A lo anterior se añade una reforma financiera para expandir el mercado en este sector (nuevo financiador del gobierno), la cual tuvo las siguientes características:
\begin{itemize}
    \item Nuevos instrumentos financieros con vencimientos más largos, con el fin de mejorar las transferencias entre agentes económicos en el tiempo.
    \item Reforma regulatoria a intermediarios financieros para incentivar su permanencia.
    \item Privatización de la banca comercial, que por breve tiempo fue nacionalizada.
\end{itemize}
Con la reforma y las acciones del banco central eventualmente disminuyó la incertidumbre, provocando así una expansión del crédito en el sistema financiero.

\section{Política económica de concertación de precios}
Uno de los compromisos del Pacto fue eliminar la inflación inercial, por lo que era necesaria una política restrictiva de demanda agregada, por ello se implementó control de precios en mercados altamente concentrados y estratégicos.

La economía entró en deflación, por lo que los empresarios bajaron precios ya que estos eran muy altos, lo anterior podía provocar una guerra de precios; para resolver esto los empresarios acordaron una disminución generalizada de los precios en un 3\% ante la presencia del gobierno, aquí se observa la cooperación entre empresarios y gobierno.

\section{Políticas comercial y cambiaría}
Fue parte importante del pacto en el cambio estructural de la economía \footnote{Ver el sector exportador de hoy día}.
Se redujeron los aranceles y los permisos para importar con lo cual aumentaron las importaciones y la balanza comercial presentó un déficit elevado, de lo cual Aspe comenta que esto difiere del déficit anterior en que antes este déficit era a causa de alto gasto público, siendo que ahora representaba un aumento en la inversión extranjera que teóricamente generaría el capital necesario para aumentar la productividad y las exportaciones en un futuro.

\section{Inflación, empleo y producción}
El impuesto inflacionario daña principalmente a la clase trabajadora. La estabilización de la inflación, los salarios y las tasas reales de interés fueron los factores que Aspe considera como claves para explicar la reactivación del consumo y en consiguiente de la economía.

En particular la inflación disminuyó de 159.2\% a 18.5\% de 1987 a 1991, donde es relevante las expectativas de la misma, reducidas debido a la congruencia de las políticas económicas; el crecimiento también se recuperó aunque a un nivel muy bajo de 1.3\% en 1988.

Las importaciones también fueron relevantes, ya que estas aumentan la eficiencia microeconómica, es decir, de las empresas. Además Aspe menciona, que la liberalización de la económica es la razón por la cual la inversión y el mercado interno se recuperaron.

En política salarial esta antes era indexada con la inflación ex-ante (que ya había pasado), pero ahora se realiza con la inflación esperada, con esto Aspe buscaba eliminar la posibilidad de fracaso del programa de estabilización; eventualmente se recuperó el salario real y el empleo formal a niveles de 1987, lo cual se refleja en los datos de trabajadores asegurados por el IMSS.

Aspe atribuye el rompimiento de la inercia a la confianza en el Pacto económico, reflejada en las acciones de libre mercado del gobierno, donde resalta la privatización; tanto la inflación como las tasas de interés continuaron disminuyendo hasta tiempos recientes.

\section{Conclusiones de Aspe}%Aquí podemos ver que pensaba Aspe del quehacer económico, es importante ya que esto guió sus acciones en la formulación de las anteriores políticas y por ende de la economía mexicana.
El PECE es reflejo del pensamiento económico de Aspe, donde resaltan los siguientes puntos que surgen de su estudio y experiencia: Aspe no piensa dejar que la economía caiga más, esto por la historia de las crisis latinoamericanas, y ve como actor fundamental al gobierno de instituciones democráticas y autónomas \footnote{recordar que Aspe propone la creación del INEGI (organismo de estadística autónomo)}. Además cree en el estudio y la preparación para el ejercicio público. Parece ser estructuralista, y que no cree en el keynesianismo de políticas de corto plazo. Ve la estabilización y el cambio estructural profundo como aspectos que van de la mano.
Aspe resume el orden temporal de la estrategia de estabilización donde primero considera que debe de tener un buen fundamento teórico: corregir las finanzas, reducir gastos excesivos, continuar con privatización de empresas pequeñas y posteriormente las grandes (para venderlas más caras con la estabilización).
Hace enfoque en el papel que tiene la confianza en las acciones del gobierno, la cual se reafirma con la congruencia; además de que cree necesaria la coordinación entre empresarios, gobierno y obreros (también considera a los extranjeros).

\section{Conclusión Personal}
Es claro que la política fue de estabilización, aunque queda en tela de juicio su efectividad. Jesús Silva-Herzog menciona que la estrategia de estabilización no fue la mejor estrategia, dado que el país en realidad necesita un enfoque en el crecimiento para mejorar la calidad de vida de las personas. Por otra parte, una de las críticas al neoliberalismo mexicano ha sido la privatización y el enfoque libre-mercado, pero por los comentarios que hace Aspe al final me parece que como humanos nos enfocamos en una serie de aspectos cuando hacemos algo y dejamos otros a un lado involuntariamente o por prioridades; en este caso la estabilización parecía razonable debido al contexto de fuga de capitales y desbalances en la balanza de pagos. Lo anterior deja un aprendizaje para las futuras políticas para enfrentar las crisis inherentes al capitalismo \cite{AjusteMacroAspe, Silva-Herzog-Opinion-Desarrollo}.
