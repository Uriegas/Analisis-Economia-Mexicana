\textbf{Reporte.}

Los resultados del 2019 son previos a la pandemia por Covid 19, y muestran una tendencia en la economía mexicana que no continuó debido a la crisis actual, en este reporte se mencionan brevemente las tendencias en los establecimientos económicos en el año 2019.

Primeramente, se considera establecimiento económico solamente a aquellos que se encuentran en fijos a la tierra, es decir, tienen un local, por lo que los puestos ambulantes no se consideran como establecimientos. La metodología para recopilar los datos es por medio de encuestas y se pueden agregar los datos para representar información municipal, regional o nacional, de allí su importancia.


En el periodo 2014 - 2019 la economía mexicana presentó un incremento en la productividad y población empleada formalmente. Los establecimientos, por su parte, aumentaron en 719,155; siendo ahora 6,373,169 establecimientos.

A su vez la composición de la producción cambió, el sector servicios, comercio y manufactura crecieron, mientras que se observa una reducción importante de la minería, que pasó de representar el 16.8\% del valor agregado total en el 2014 a representar solamente 9.5\% en el 2019. Por otra parte, la manufactura sigue siendo una parte importante de la economía, representando casi una tercio de esta (32\%).

En cuanto a la importancia de estos sectores por entidad federativa, en la mayor parte de los estados se observa que el sector manufactura representa una parte importante de la generación de valor agregado, mientras que los estados de Oaxaca, Chiapas y Guerrero tienen como principal sector al comercio. Por otro lado, la CDMX es la entidad con una mayor relevancia del sector servicios, que representa el 73\% de su economía.

México se caracteriza porque los establecimientos son en su mayoría micro-negocios, representando el 94.9\% del total de establecimientos (se considera micro-negocio a todo establecimiento que emplea de 0 a 10 personas).
Solamente el 0.2\% de los establecimientos son grandes (más de 250 personas), aún así, estos generan más de la mitad del valor agregado en el país, y emplean a casi un tercio de la población. Lo cual contrasta con la poca generación de valor agregado de los establecimientos micro, aunque emplean poco más de un tercio de la población.

Los establecimientos grandes son los que principalmente capacitan a su empleados, siendo el 59.1\% de establecimientos que capacitan a sus empleados, aún así, la rotación de personal es alta en estos establecimientos.

En general todos los establecimientos presenta ciertos problemas generales para operar o crecer, entre estos destacan la inseguridad pública, los altos gastos de los servicios básicos y los altos impuestos, aunque este último afecta principalmente a los grandes establecimientos. 

En general los grandes establecimientos presentan un mayor uso de las tecnologías de la información, aunque las PYMES no se encuentran muy rezagadas en este aspecto.

En cuanto a la remuneración del trabajo, el personal remunerado ha aumentado un 5.9\% en el periodo 2014-2019, aunque esto es un buen indicador, hay que considerar que este aumento de personal fue mayor que el aumento de remuneraciones (4.1\%), es decir, se esta empleando a más personas, pero no se le esta remunerando más o tan siquiera lo mismo, en consecuencia el indicador de remuneraciones por persona ha disminuido, lo cual ha sido una constante desde el 2009, aunque resalta que esta disminución en las remuneraciones por persona se dan principalmente en los establecimientos grandes, que presentaron una disminución del 3\% del 2014 al 2019, en comparación de la disminución del 0.1\% de los establecimientos micro durante el mismo periodo.

En cuanto a la composición del empleo, la composición del mismo entre personal remunerado, propietario no remuenrado o personal outsourcing no ha cambiado mucho, aunque si se observa una tendencia a un mayor outsourcing (16.6\% a 17.3\% en 2014-2019) y un aumento del personal remunerado (56.5\% a 59.9\% en 2014-2019), disminuyendo entonces los propietarios no remunerados.

Por último, la informalidad de los establecimientos, que cuentan con varias características, siendo las principales que no realizan contribuciones patronales al régimen de seguridad social y que emplean menos de 5 personas. Ha sido constante la informalidad en los establecimientos mexicanos, sobre todo en los de tipo comercio, que son los que menos valor agregado generan pero que emplean a la mayoría de la población. El 62.6\% de los establecimientos son informales, y solamente generan el 3\% del valor agregado, estos se encuentran principalmente en los estados de Guerrero, Oaxaca representación en la CDMX \cite{Censo2019}.
