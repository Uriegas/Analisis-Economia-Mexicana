La crisis que trajo el Covid 19 tuvo altos efectos tanto en el entorno internacional como en el nacional, esta situación se diferencia de la crisis del 2008 en que la actual fue un cambio en el sector real que afecto al sector financiero, mientras que la del 2008 fue al revés. A continuación se presenta el estado actual (reapertura y post-covid) de las variables macroeconómicas más importantes.

\textbf{Entorno internacional.}

La pandemia ha traído restricciones a la movilidad que se traducen en una disminución en la producción y cambios en el consumo, es decir, la demanda.
En general se espera una recuperación moderada de la actividad económica para todo el año 2021, esto propiciado por la aparición de vacunas contra el Covid-19 y los esfuerzos de los Estados por aplicar medidas contra cíclicas para estimular la actividad económica.
Aunque ya existen diferentes vacunas aún así existe bastante incertidumbre respecto a las mismas y por lo tanto del panorama económico, ya que existen retrasos en su producción, distribución y aplicación, así como diversos problemas técnicos como el almacenamiento en frío de las mismas, la calidad, los efectos secundarios y su efectividad, por mencionar algunos.

\textbf{Situación de la economía nacional.}

Con lo mencionado sobre el entorno internacional, en el corto plazo se percibe incertidumbre, aunque en el largo plazo se espera una recuperación de la actividad económica.
Aún así, las perspectivas de crecimiento económico actuales son más pesimistas que en el trimestre anterior, esto derivado de los retos que se han visto en la actualidad con la vacunación, la reapertura y el suministro eléctrico en el norte.
El Banco de México prevé que la recuperación económica a los niveles de producción pre-covid llegue a finales de este año o de 2022, en el escenario más pesimista y el más positivo, respectivamente.


\textbf{Sector externo.}

Recordando que México tiene a Estados Unidos como agente económico principal, se han presentado cambios en la demanda americana, y se prevé que sigan habiendo, lo cual ha generado volatilidad en los precios de las exportaciones mexicanas, principalmente en los alimentos mexicanos.
A su vez, se espera un superávit en la balanza de pagos, aunque solamente durante el periodo de recuperación y pandemia, esto para el año presente, mientras que para el 2022 se espera un déficit, como un retorno a los valores de la balanza comercial pre-covid.

\textbf{Empleo.}

Se espera un aumento del empleo, impulsado por la recuperación económica. Con una variación entre 250 y 570 mil empleos, percepción más positiva que la del informe pasado.

\textbf{Inflación.}

En cuanto a la inflación existen choques entre la oferta y la demanda propiciadas por los cambios estructurales entre las mismas que trajo el confinamiento.
Del lado de la oferta se observa una destrucción de ciertas cadenas de suministro, tanto globales como nacionales, así como un aumento de los costes de producción, y en general una reducción de la producción total.
De lado de la demanda, existió resignación del gasto de los hogares, resaltando el aumento de la demanda de alimentos, así como un mayor uso de servicios, algunos de estos cambios se mantendrán en las futuras tendencias de consumo, mientras que otras habrán sido solo temporales.
Actualmente existen presiones a la baja de la inflación, aunque con volatilidad, a pesar de que disminuyó la inflación en el trimestre anterior, se observó un aumentó de la misma en el mes de febrero de 2021, esto muestra la volatilidad actual, en la cual participan tanto los cambios en la demanda, oferta, el sector externo y eventos como el Buen Fin.
Ante este entorno el Banco de México decidió reducir sus tasas de interés, con el fin de cumplir su objetivo por ley de mantener la inflación estable a un nivel de 3\%.

\textbf{Riesgos para el crecimiento.}

Se observa principalmente una alta incertidumbre con respecto al panorama global y nacional, y los efectos que tendrá en la economía nacional. La información actual muestra un debilitamiento de la economía nacional, en todas las áreas: empleo, oferta, consumo, gobierno.
Algunos de los escenarios que más incertidumbre generan son el prolongamiento de la pandemia, mayor volatilidad en los mercados financieras, secuelas sobre la economía, reducción de la calificación de deuda de PEMEX, desacuerdo en la subcontratación en México, reducción de la inversión.
Aunque de lado positivo genera incertidumbre el plan de recuperación económica de las principales economías, desaparición de la pandemia en el corto plazo, recuperación de la demanda americana (lo cual puede aumentar las exportaciones mexicanas) y la entrada en vigor del nuevo tratado de América del Norte.\\

El panorama actual es positivo al largo plazo, las campañas de vacunación, económicamente, generaran una mayor movilidad que dará lugar a un aumento en el consumo, producción y comercio. En resumen, se prevé un escenario de recuperación económica; aún así persiste un clima de incertidumbre y una creencia en hacer más flexibles a las instituciones, esto se refleja en los programas de estimulo en gran parte del mundo, resaltando la decisión de la FED de mantener tasas de interés bajas hasta el 2024 y la decisión de aplicar el estimulo fiscal más grande de la historia de Estados Unidos; actualmente se esta generando un debate sobre los posibles efectos que esto traerá sobre la economía tanto americana como global, sin duda esto afectará en la relación de dependencia México - Estados Unidos.

\cite{Banxico4toTrismestre}