En este trabajo se presentan las 4 etapas del desarrollo del capitalismo en México, en cada una de ellas se presentan los conceptos más importantes, así como un breve análisis con enfoque en el PIB, su varianza y el ingreso.

\textbf{Transición al capitalismo (1860-1939)}

Se subdivide en 2 periodos caracterizados por la inestabilidad:
\begin{itemize}
    \item 1860 - 1910. El ambiente político era de reestructuración del país, se implementó el modelo primario-exportador aprovechando la gran oferta de mano de obra mexicana, fue un periodo de inestabilidad, lo cual se ve reflejado en las fluctuaciones del PIB en el periodo 1896-1903 y 1926-1933 que el Estado no controlaba.
    \item 1910-1934. Fue un periodo de mayor inestabilidad política y económica (varianza del PIB 1925-1939 = 49\%), tanto globalmente como nacionalmente, después de la Revolución se generó un auge económico, del cual se desarrollaron industrias como la banca, el sector público y el sector petrolero.
\end{itemize}

\textbf{La primera acumulación industrial (1940-1954)}

Este periodo se caracterizó por un desarrollo económico continuo y estable, propiciado en primera instancia por la 2da guerra mundial que aminoró la competencia para las exportaciones mexicanas.
La acumulación de capital se \textit{financió} con el deterioramiento del salario y un aumento en la inversión privada, este último relacionado con el aumento de pequeñas y medianas empresas, las cuales fueron las principales promotoras del desarrollo industrial del país. Además de esto, la inversión incluía la importación de capital, es decir, maquinaria que sería útil para aumentar la productividad.
A pesar de la estabilidad económica generalizada, existieron periodos de inestabilidad provocados por movimientos económicos en el mercado global (guerras).
El crecimiento impulsado por la guerra se constata con la desaceleración económica de 1953 al término de la guerra de Corea, año en el que el crecimiento del PIB fue de 0.3\%, la producción industria y la construcción cayeron (1.1\% y 7.7\%).

\textbf{La transición al desarrollo oligopólico (1955-1961)}

Aunque el crecimiento económico se mantuvo en el mismo nivel del periodo pasado (5.9\%) durante este periodo se presentaron cambios de tipo estructural, en el ingreso, la producción y el consumo.

Primeramente, el periodo presentó niveles bajos de inestabilidad, siendo la variación del PIB de 4\% (en el periodo pasado fue de 9.9\%), lo cual muestra resultados de la llamada \textit{estrategia del desarrollo estabilizador}.


El ingreso se concentró, es decir, aumentó la riqueza de los ricos sin un aumento de la riqueza de los pobres; a esto aunado el hecho del aumento del ingreso de capital extranjero en la fabricación de bienes duraderos que satisfacían la demanda de esta concentración, aún así la inversión privada se redujo de 11.5\% a 3.3\%, debido a que la concentración excluyó a el resto de la población.

La disminución de la inversión privada se compensó con el aumento de la inversión pública, apareció la \textit{expansión no planeada}, la cual describe la adquisición de empresas privadas por el Estado; así como un aumentó en la burocracia. El aumento de la inversión pública, la estabilización y el desarrollo del sector financiero provocaron una disciplina financiera del Estado, así como una mayor conexión con el capital bancario y transnacional (americano).

\textbf{El desarrollo oligopólico (1962-1977)}
Este periodo, al igual que el primero, se subdivide en 2 partes:
\begin{itemize}
    \item 1962-1970. Rápido crecimiento económico, baja inflación y estabilidad cambiaria.
    \item 1971-1977. Crisis económica, hiper inflación e inestabilidad cambiaria.
\end{itemize}

EL periodo fue parte del milagro mexicano (PIB de 7.6\%), marcado por un crecimiento económico impulsado por el capital extranjero y la inversión pública, desarrolladas en el periodo pasado; el Estado creó nuevos campos de producción y el capital extranjero fabricó bienes duraderos para la creciente la población que concentraba el ingreso. Este desarrollo se observó primordialmente en las grandes ciudades.

Aún así este desarrollo presenta desigualdades que se evidenciaron con las crisis posteriores del segundo subperiodo. El sector agrícola se dividía en los pequeños productores y las grandes empresas, estas últimas aprovechaban los bajos salarios y el ambiente proteccionista para realizar inversiones en las nuevas tecnologías que aumentaron la productividad y por ende las ganancias y la posición de monopolio en el mercado.
La concentración tanto del ingreso como de la producción generó un sector privilegiado caracterizado por una ideología \textit{consumista}, creando así un mercado de \textit{élite} (productos caros para los capitalistas), el cual se satisfacía con bienes financiados con capital extranjero, situación que impulsó el desarrollo del sistema financiero.

\textbf{Conclusiones}

Observando la gráfica del desarrollo del PIB y su varianza (nivel de inestabilidad) se observa una tendencia a un mayor crecimiento económico con más estabilidad, siendo el periodo de 1925-1939 el más inestable, esto por los retos de la inestabilidad de los mercados mundiales, situación que desarrolló el mercado interno (aunque desigualmente), los primeros periodos parecen ser momentos inestables que intentaron establecer las bases institucionales para un mejor desarrollo en el futuro, mientras que los últimos 2 periodos reflejan ese desarrollo estabilizador, aunque no se tomó en cuenta el rumbo que llevaba ese desarrollo, que, como ya se mencionó, fue impulsado por capital extranjero hacia una mayor concentración en la producción (oligopolios) y el ingreso (élite), lo cual no solo se observa en el caso mexicano, sino en gran parte de Latinoamérica y tiene efectos hasta el día de hoy.
El periodo que me parece es el punto de inflexión es el de 1940-1954, ya que la importación de capital(27\% del PIB) fue clave para el desarrollo posterior, así mismo definió la nueva forma de desarrollo con el aumento del capital extranjero (sin olvidar la importancia de la inversión privada doméstica 58.19\% de la inversión total), la inversión pública y la concentración del ingreso, la cual terminó formando los oligopolios y la élite en los periodos posteriores; además de ser el punto en el que el PIB pasa de 1.5\% a 5.8\%, y la inestabilidad disminuye de 49\% a 9.4\%.
A lo largo del desarrollo mexicano se observa una intervención constante del Estado, al principio en el establecimiento de las reglas e instituciones que harían funcionar el país (mercado y Estado), posteriormente con la inversión pública (42.8\% en 1940-1954) que permitió el desarrollo de la industria petroquímica, entre otras, y del sistema financiero, el cual dio lugar a una mayor inversión extranjera que satisfizo el mercado de \textit{élite} \cite{PeriodizacionMexico}.