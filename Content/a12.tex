% Elabore un reporte en el que describa las fuentes de crecimiento económico en la década de los años cincuenta, así como las características de la política económica instrumentada en esos años.

Durante la de cada de los 50's México experimentó un acelerado crecimiento económico basado en.
Capacidad de la economía de mantener un ritmo de crecimiento adecuado.
El gobierno en los 40's realizó un fuerte gasto en infraestructura que tuvo como resultado un aumento en la inversión privada reflejado en el crecimiento de los diferentes sectores, a su vez el PIB per cápita creció a un ritmo de 3\% durante el periodo, añadiendo el crecimiento poblacional dadas las mejores condiciones sanitarias.
Normalmente se dice que la inversión pública estimuló a la economía, pero ¿de qué manera? Aquí se busca explicar las fuentes de crecimiento a un nivel mayor de profundidad.

\section{Fuentes de crecimiento económico}
\subsection{Ahorro interno}
A partir de la moratoria de 1913 y los periodos de guerra México perdió acceso al crédito externo, lo que lo llevó a buscar otras formas de financiarse, incrementando su ahorro interno mediante el desarrollo de la banca mexicana; además se existía cierta cautela a la hora de solicitar crédito externo, debido a las experiencias pasadas.
El ahorro externo llegó a ser el 4.9\% del ahorro total, mientras que el Estado no aumentaba su déficit (no se endeudó) llegando a representar solamente el 0.3\% del PIB. Junto con el desarrollo mencionado, se generó un ciclo entre ahorro e inversión que generó crecimiento.

\subsection{Crecimiento igualitario entre sectores}
De cierta forma la mayoría de los sectores crecieron a ritmos similares (agricultura 4.4\%, manufactura 7\%, petrolera 7.8\%, eléctrica 9.1\%), lo cual difiere de las estrategias futuras basadas en el sector petrolero o el sector manufacturero exportador.

\subsection{Aumento de la productividad}
La inversión privada (91.6\% de la inversión nacional) realizó cambios en la relación del capital-trabajo en varias industrias, esto se refleja en el aumento de productividad de la mano de obra de 24.1\% durante el periodo, en el flujo de capital que creció 4.5\% anualmente, y en el aumento de los salarios reales de 2.2\% anual. Todo esto recordando que la inversión privada fue posible gracias a la inversión pública en infraestructura y las condiciones externas: aumento de la demanda externa debido al periodo de post-guerra.

\subsection{Mercado interno protegido (Proteccionismo)}
El mercado de bienes de consumo se protegió de forma que no se permitió la entrada de extranjeros, lo cual fortaleció a la industria interna, aunque dejaba a su merced a los consumidores mexicanos. A su vez la industria también se dedicó a la producción de bienes de exportación, necesario para la generación de divisas las cuales servían para la compra de insumos y bienes de capital necesarios para producir bienes de consumo. 

\subsubsection{Desarrollo de la banca}
Con el crecimiento de la producción y el mercado nacional, los servicios financieros fueron en aumento, los cual hacen más eficiente la transformación de ahorro a inversión mediante la facilitación del crédito.

\subsection{Crecimiento del Mercado interno}
Más que una sustitución de importaciones lo que en realidad ocurrió (según los datos del autor) fue que el mercado interno creció, esto por diversas razones, entre ellas el aumento poblacional y los cambios en la forma vivir (urbanización). El mercado interno era satisfecho por casi el 95\% de la producción nacional y en un ambiente proteccionista el crecimiento del mercado solo hizo que creciera más la producción nacional.

\section{Características de la política económica}
\subsection{Contrarrestar fluctuaciones del exterior}
Existieron diversos cambios en las reservas internacionales provocadas por variaciones en la demanda externa (Estados Unidos principalmente). El aumento en las reservas provoca un aumento en la oferta monetaria y por lo tanto inflación, para contrarrestar esto el banco central vendió oro, plata y bonos para absorber liquidez. A pesar de la inflación generada por las reservas, estas tuvieron un efecto positivo al servir como divisas para el gasto en importaciones, el cual aumentó 48.9\% en 1951.
El caso contrario sucedió con la caída de la demanda externa que deterioró los términos de intercambio en 7.5\% provocando una posible deflación contra la que se actuó liberalizando mercados, aumentando el gasto público y obteniendo crédito externo del FMI.
\subsection{Estabilidad de precios}
Se logró por el control de la demanda agregada, aplicando el enfoque keynesiano, se devaluó la moneda, de 8.65 a 12.50 pesos por dólar, para estimular las exportaciones y aumentar las reservas (dada su previa reducción), a esto se aplicó un impuesto de 25\%, lo que aumentó la recaudación fiscal; por otra parte esta medida provocó una perdida de capitales contrarrestada con expansión de crédito a los bancos privados. Al final esta serie de acciones tan específicas generaron un periodo de crecimiento de 3 años.
Además de políticas monetarias (devaluación de la moneda), se usó política fiscal para aumentar la recaudación fiscal y financiar el gasto público; adicionalmente, el gobierno hizo llamados a los empresarios a no aumentar precios; todo ello sirvió para reducir las expectativas de inflación y mantener preciso estables.
\subsection{Proteccionismo}
Políticas como las anteriores se usaron para proteger y promover la producción interna. En un principio, como ya se mencionó, el Estado limitó las importaciones de bienes de consumo para estimular la producción nacional y a su vez permitiendo la importación de insumos; después, con la mayor competencia internacional en el mercado de bienes, el país compensó el déficit comercial con servicios como el turismo, remesas e inversión externa (a corto plazo principalmente).

\section{Conclusión}
Es un hecho que los 50's fueron un periodo de bonanza, sus efectos se atribuyen al gasto público, proteccionismo  y sustitución de importaciones aunque con inflación. El autor presenta un análisis más detallado con un enfoque en los factores externos e internos macroeconómicos, con énfasis en la política económica.
Atribuyendo la inflación no al gasto público, sino a ...
la sustitución de importaciones no a una política o estrategia, sino al creciente mercado mexicano.
Las condiciones externas fueron buenas aunque cambiantes, resalta el comportamiento estratégico del Estado en afrontar estos cambios y a su vez propiciar el crecimiento. me parece que es necesario realizar un análisis económico profundo del país de forma que se tomen decisiones estratégicas que respondan a los problemas afrontados conservando los objetivos de desarrollo \cite{CrecEcoSano}.