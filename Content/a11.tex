%Elabore un reporte en el que comente las características de las finanzas públicas y de las políticas monetaria y financiera como instrumentos para favorecer el crecimiento. Finalmente, considere el papel y la política de los organismos e instituciones públicas para alcanzar el propósito de crecimiento.
% pp. 265-283.

\section{Finanzas Públicas y el desarrollo}
\subsection{Gasto Público}
Cambios en la estructura del gasto público, entre 1940 y 1954 creció 8\% anualmente. En aspectos económicos creció 17\%, mientras el social y administrativo se redujeron debido a la reestructuración de la relación del estado con estos 2 (5\% y 2\% respectivamente). Para 1954 61\% de los egresos representaban aspectos económicos.

Este gasto en lo económico buscaba la modernización de la economía mediante mecanismos de mercado apoyando la acumulación privada (sector industrial principalmente).
El periodo de Ávila Camacho sentó las bases para el expansionismo de Miguel Alemán, aumentó pausadamente los egresos totales, el gasto anual creció a una tasa de 2.8\%, aunque aún persistía la idea de un presupuesto equilibrado.

La expansión comienza en el sexenio de Miguel Alemán, el gasto anual creció a una tasa de 17.3\%, este gasto fue en inversión pública para modernizar la estructura económica que funcionaba como un generador de condiciones para propiciar la acumulación privada (infraestructura e insumos básicos). Aumentó la relación inversión/PIB de 7.6\% en 1940 a 15.8\%, es decir, el país en general gastaba más en inversión que en consumo. El gasto tenia un sesgo hacia lo industrial (comunicación y transporte principalmente).

\subsection{Sistema tributario}
 EL gasto se financió con deuda y aumento de las tasas tributarias(aumento de 7\% anual). Los impuestos sobre exportaciones aumentaron mientras que los de importaciones disminuyeron, cayó la carga fiscal sobre el sector industrial, aumentaron los impuestos sobre la renta y ad valorem (Cambio de la estructura del sistema tributario). En resumen, se usó el sistema tributario para promover el crecimiento económico.

\section{Política monetaria y financiera}
Se cambió la estructura del financiamiento del déficit, cayendo la participación del Banco de México en el mismo (de 90\% en 1940 a 78\% en 1954) y con la apertura de un mercado de bonos estatales. Los objetivos de las autoridades monetarias fueron: fomentar la producción, controlar la inflación y la estabilidad económica.

Se divide la política monetaria en los siguientes periodos:
\begin{itemize}
    \item 1940-1945. El banco de México vendió plata y oro al público para contrarrestar el aumento de las reservas internacionales. Los bonos gubernamentales se volvieron más líquidos (fueron adquiridos por bancos privados) pasando de 35M a 153M de 1943 a 1945.
    \item 1946-1949. Se deterioró la balanza de pagos restringiendo los planes de industrialización. Disminuyeron las divisas y el crecimiento PIB. Se buscó aumentar las reservas internacionales a través de la disminución de venta de oro y plata, acceso al crédito externo, retirada del mercado cambiario y la colocación de valores.
    \item 1950-1954. Fue un periodo de diversos ajustes para contrarrestar los efectos del incremento del gasto público, además disminuyó la participación del Banco de México en el financiamiento del gasto público.
\end{itemize}

Mejoró la posición financiera de México debido a convenios firmados durante la segunda guerra mundial. El financiamiento del gasto fue cambiando, de tal manera que el ahorro externo se volvió importante. Finalmente, al final del periodo la transferencia de recursos hacia el sector privado refleja la disminución de la participación del Estado en el desarrollo.

\section{Empresas públicas}
Aumentaron los organismos descentralizados de 29 a 123, así como la inversión en los mismos de 33\% a 54\% (1940-1954). La inversión en el sector público fue principalmente en insumos, comunicaciones y servicios financieros; lo que, en conjunto, generaron las condiciones para el desarrollo del sector privado. Las empresas paraestatales dejaron de depender de sus propios recursos y pasaron a usar crédito externo para financiarse, pasando de representar el 3.81\% en 1943 a 28.1\% en 1954.

Además, estas promovieron la reformación del sector agrícola mediante la tecnología, apoyo financiero y con obras de infraestructura tales como las presas y generación de electricidad, beneficiando así a los productores locales.
Las empresas estatales también apoyaron con los insumos básicos, en los siguientes sectores:
\begin{itemize}
    \item Electricidad. Sector dominado por el capital extranjero, con la llegada de la CFE paulatinamente esta fue ganando terreno en la venta de energía, no por mandato, sino por cooperación entre pequeñas empresas eléctricas e inversión. La energía se vendía principalmente a fabricas (industria) lo que representó un apoyo del Estado a la industria.
    \item Petróleo. Hubo una caída en la producción y exportación debido a la disrupción ocasionada por la expropiación petrolera, lo cual se compensó con el aumento de la demanda interna. Progresivamente aumentó la inversión de PEMEX, en parte al apoyo de otras empresas públicas financieras como NAFINSA.
    \item Minera. La participación en la producción de bienes mineros se debe a la política de sustitución de importaciones, sobretodo por la escasez generada por la guerra. Se creó Altos Hornos de México, nuevamente con financiamiento de NAFINSA, eventualmente surgieron más empresas en este ramo, de las cuales el gobierno adquirió una participación; con lo cual se aumentó la producción.
    \item Comunicaciones. Con la creación de empresas ferrocarrileras y una dedicada a la fabricación de camiones a Diesel.
\end{itemize}
Además de estos hubo más sectores en los que intervino el Estado, ya fuera creando empresas, teniendo participación en las mismas, aportando insumos o proporcionando infraestructura o control de precios.

\section{Conclusión}
Sobresale el compromiso del gobierno con el desarrollo económico por medio de una intervención directa en la economía (gasto público) mediante un estudio caso por caso de las necesidades y oportunidades de los sectores y regiones económicas, resalta la importancia del sector financiero permitiendo el desarrollo de industrias que requerían grandes inversiones en capital; todo ello fue posible debido a una buena situación política y social, aunque con el tiempo fue deteriorándose. Me parece que fue un periodo que deja como lección que el gasto público puede ser beneficioso y no precisamente ineficiente, esto me recuerda a la intervención del gobierno chino en la economía, que puede entenderse como una intervención caso por caso, existiendo sectores con gran participación estatal, mientras que otros están predominados por particulares. Por otra parte, deja como lección que no debe dejarse de lado las problemáticas sociales y políticas que muchas veces son más complicadas que las económicas y tienen una alta repercusión en esta \cite{EstadoyDesarrolloUNAM}.