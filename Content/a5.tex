
El empleo sufrió varios cambios a partir de la crisis por Covid 19, aquí se presentan las definiciones y datos de las principales variables macroeconómicas en el último trimestre de 2020, con el fin de entender los datos básicos de empleo que reflejan la situación actual; finalmente se provee la opinión del autor sobre el panorama del mercado laboral.

\begin{itemize}
    \item \textbf{Población Económicamente Activa (PEA).} Son las personas mayores de 15 años, ya sea que estén ocupados o desocupados, es decir, trabajando o en busca de. Se observó una disminución de la PEA en un 3\% de 2019 a 2020, quedando un total de 55.9 millones de personas dentro de la categoría PEA, es de resaltar que la reducción fue mayor para las mujeres con un 5\%, mientras que para los hombres fue de 1.7\%.
    \item \textbf{Población No Económicamente Activa (PNEA).} Toda aquella persona que no esta trabajando y no esta en busca de un trabajo, en esta categoría entran los estudiantes, pensionados, amas de casa, etcétera. Dado que o se es PEA o PNEA, la disminución en la PEA provocó un aumento de las personas que ya no trabajan y no están buscando trabajo; con un total de 41.3 millones de personas en esta categoría; de todas estas personas el 21.6\% aceptarían un trabajo si se les ofreciera.
    \item \textbf{Tasa de Participación Económica.} Es el porcentaje de la PEA con respecto a la población mayor a 15 años. Este indicador fue de 57.5\% para finales de 2020, aunque con una reducción del 2.9\% de 2019 a 2020.
    \item \textbf{Población Ocupada.} Toda aquella persona que en la semana de la entrevista realizó al menos 1 hora de trabajo, incluyendo a aquellos que por alguna situación especial no trabajaron pero si tienen trabajo. Este tipo de población compone el 95.3\% de la PEA, con un 53.3 millones de personas; de toda la población ocupada el 68.6\% esta a sueldo, mientras que el 22.8\% trabaja a cuenta propia. Es de notar que de 2019 a 2020 la cantidad de personas no remuneradas cayó en un 14.3\%.
    \item \textbf{Población Ocupada por sector de actividad.} División de esta población entre sectores. La mayor parte de los ocupados se encuentran en el sector terciario, capturando el 61.9\% de los ocupados. Aquí los mas relevante es la disminución en el sub-sector restaurantero y de alojamiento, que tuvo una reducción de 17.4\%, esto evidentemente por la pandemia; por otro lado, el el gobierno tuvo un aumento del 6.1\%.
    \item \textbf{Población Subocupada.} La población que oferta más empleo que la demanda actual, es decir, tienen necesidad de trabajar más pero no hay tanto trabajo. Esta población pasó de 4.3 a 8.1 millones, un aumento del 88.3\%; los trabajadores asalariados representan la mitad de los subocupados, mientras que otro 39.1\% se compone de los trabajadores a cuenta propia. El sector que mayor aumento de subocupados tuvo fue el sector servicios con 7.2\%, siendo que el 65.8\% de los subocupados se encuentran en este sector. Esto quiere decir que la población en este sector no solo presentó una alta disminución, sino que parte de aquellos que quedaron no están trabajando todas las horas que desean.
    \item \textbf{Población Desocupada.} En búsqueda de trabajo, pero que no están trabajando en ninguna unidad económica actualmente. Comprenden el 4.6\% de la PEA. Poco menos de la mitad de esta población se encuentra entre los 25 y 44 años (47.3\%), se observa que esta población fue la que más aumentó durante el 2020, mientras que la que se redujo fue la población de 15 a 24 años en un 5.8\%. En cuanto al tiempo en situación de desocupación las personas tienden a durar poco tiempo aquí, dado que en México no hay seguro de desempleo; por otro lado durante el 2020 se observó un aumentó del 9.2\% de los desocupados que tienen entre 6 meses a 1 año en esa situación.\\
    
    \textbf{Tasas Complementarias.}
    \item \textbf{Tasa de Presión General.} Población desocupada más la población ocupada que busca trabajo respecto a la PEA (Des + Ocu / PEA), expresa el porcentaje de personas que buscan trabajo (demanda laboral). A finales de 2020 fue de 8.3\%, con un aumento de 1.6\% con respecto al 4ª trimestre 2019. 
    \item \textbf{Tasa de Informalidad Laboral.} Ocupados vulnerables mas aquellos que sus unidades económicas no los reconocen. Estos representan el 55.6\% de los ocupados, con una disminución de 0.6\% de 2019 a 2020.
    \item \textbf{Tasa de Ocupación en el Sector Informal.} Ocupados trabajando con una unidad económica sin estar constituida como empresa. Fue de 27.9\% de los ocupados, con un aumento de 0.5\% con respecto a 2019.
    \item \textbf{Tasa de Condiciones Críticas de Ocupación.} Población trabajando menos de 35 horas, mas las que trabajan más de 35 horas pero con menos del salario mínimo, mas las que trabajan más de 48 horas con menos de 2 salarios mínimos. Esta tasa fue de 23.5\%.
    \item \textbf{Tasa de Subutilización de la Fuerza de Trabajo.} Población desocupada, subocupada o no económicamente activa de la PEA mas la PNEA disponible para trabajar $(Des + Sub + PNE) / (PEA + PNEA_{Dis})$. En general fue de 30.3\%, mientras que las ciudades con mayor subutilización son Coatzacoalcos (45.7\%) y Ciudad de México (42.1\%), mientras los de menor subutilización son Saltillo (15.1\%) y Tijuana (19.8\%).
\end{itemize}

\textbf{Comentario Condiciones del Mercado Laboral Mexicano}

Con los datos analizados se observa en general una deterioración del empleo, se vio principalmente afectada la ocupación femenina, cierta parte de las personas ocupadas pasaron a ser desempleados, sobre todo aquellos que son jóvenes adultos, la demanda laboral ha aumentado (ver Tasa de Presión General), mientras que la informalidad no parece haber aumentado como se pensaría, aún así es difícil respaldar la veracidad de esta parte de la encuesta, con los datos recabados tal parece que buena parte de la población esta desocupada pero no esta buscando trabajo informal, sino buscando otro trabajo formal. Llama la atención que el único sub-sector que aumentó el número de personal, además del gobierno, fue la construcción, esto a pesar de que en términos de producción presentó una disminución de 20\% \cite{ConstruccionINEGI}, situación que quizá requiere de un mayor análisis.

\footnote{Todos los conceptos fueron obtenidos del glosario de la página oficial del INEGI.} \cite{ENOE2020} \cite{GlosarioINEGI}