El crecimiento económico en México ha presentado avances y retrocesos, aquí se presentan 3 grandes etapas: 1780-1870, 1871-1939, 1940-1980; el primero representando el fin de la etapa colonial y los primeros años del país, el segundo el periodo de Porfiriato y la post-revolución, y el tercero los años dorados; cada etapa se divide en 2 grandes secciones: obstáculos y factores de crecimiento; finalmente se concluye con la opinión del autor.

\section{El siglo de deterioro 1780-1870.}
\subsection{Factores}
\begin{itemize}
    \item Política de libre mercado. La cual aumento las exportaciones mexicanas, aunque generó desigualdad en un norte exportador y un sur agrícola subdesarrollado, además entró competencia extranjera la cual fragmento mercados locales.
    \item Haciendas. Funcionaron como empresas agrícolas eficientes, aprovechando sus ventajas en crédito, mercados, economías de escala, acceso a tecnologías, etcétera; se generó un grado de división de trabajo (haciendas especializadas en un bien).
    \item Iglesia. Funcionó como un banco, cobrará diezmos al producto de la tierra y usaba esos recursos para otorgar créditos en bienes raíces, lo que aumentó la acumulación de capital, aunque tenia un sesgo contra la industria, limitando su desarrollo.
    \item Conservadores. Tenían un enfoque en los avances económicos (ej. Banco de Avío), aunque dejando de lado las problemáticas sociales.
\end{itemize}

\subsection{Obstáculos}
\begin{itemize}
    \item Prolongada inestabilidad política. Tanto en los últimos años de la era colonial como en los primeros años del México independiente existió una gran inestabilidad política, en los 55 años antes del Porfiriato la presidencia cambio 75 veces de manos.
    \item Disminución de la producción minera. Principalmente de la plata, lo que provocó un éxodo de mineros españoles y consecuentemente disminución de capital financiero.
    \item No se reformó el código comercial. A pesar de que eran necesarios cambios en las políticas económicas nunca se reformaron, esto pudo haber evitado el éxodo minero y mantener la industria manufacturera (22.3\% de producto en 1800).
    \item Desigualdad. Desde la época de la Colonia México fue una sociedad con grandes diferencias sociales, fundamentalmente por las diferencias culturales, raciales y religiosas que prevalecen inclusive hoy día.
    \item Economía politizada. Las empresas dependían de su relación con el gobierno.
\end{itemize}

\section{Estado desarrollista 1871-1939.}
\subsection{Factores}
\begin{itemize}
    \item Estabilidad política. Después de la inestabilidad en el país, el gobierno tomó el lema: Orden y Progreso, considerando necesaria la estabilidad para el desarrollo del país, así como el progreso económico mediante la industrialización.
    \item Política económica enfocada en inversión privada. La política se enfocó en atraer inversión privada y reducir las barreras al comercio, por lo que se reformaron los códigos de comercio y minería, se eliminaron las barreras arancelarias interregionales y externas y se atrajo capital extranjero; esta política aumentó triplicó la relación exportaciones/PIB de México en el periodo 1870-1913
    \item Factores externos. Entre ellos se encuentran que el periodo del Porfiriato coincidió con la segunda revolución industrial y la demanda de minerales por parte de los países donde se daba esta revolución; otro factor fue la reducción del valor de la plata, lo cual favoreció a las exportaciones mexicanas.
    \item Estabilidad social pos-revolucionaria. Después de la revolución se creó el PNR y se fortalecieron las instituciones de gobierno, así se presentó una etapa de estabilidad social y política que resolvió el descontento del Porfiriato.
\end{itemize}


\subsection{Obstáculos}
\begin{itemize}
    \item Desigual distribución del ingreso. Las tierras se concentraron en unas cuantas personas, dejando a los obreros en una posición de explotación, evidencia de ello es la reducción de salarios reales en 26\% (1903 a 1910); esto aumentó la pobreza rural.
    \item Poco progreso político y social. El objetivo era convertir a México en un país industrializado, aunque se dejó de lado los avances sociales necesarios para que este fuese un desarrollo sostenido, evidencia de ello fue la hambruna y baja alfabetización.
\end{itemize}


\section{Años Dorados 1940-1980.}
\subsection{Factores}
\begin{itemize}
    \item Mayor participación del Estado. El Estado se enfocó en realizar inversión por su cuenta, a diferencia de la estrategia de la etapa pasada que era atraer inversión. El modelo funcionó muy bien, tanto así que el crecimiento fue de 6.4\% anual, aumentó la población urbana de 35\% a 66\%, la alfabetización llegó a 83\% en 1980.
    \item Proteccionismo. Contra extranjeras y se aplicó sustitución de importaciones.
\end{itemize}

\subsection{Obstáculos}
\begin{itemize}
    \item Agricultura relegada. El enfoque fue en lo urbano e industrial, dejando de lado lo rural y la agricultura, lo cual provocó un aumento de la pobreza en esta población.
    \item Baja inversión en comercio internacional. En las exportaciones existió una falta de política que apoyara el desarrollo de esta, en la sustitución de importaciones no era posible sustituir bienes de capital que requerían alto nivel de tecnología.
    \item Reformas tributarias. No fueron capaces de sustentar al gobierno, el cual incurrió en deuda externa, la cual se convertiría en un problema hasta el día de hoy.
    \item Crisis de 1976. En este tiempo aún no se conocían las reservas petroleras en México por lo que el país importaba crudo, el cual tuvo una baja de demanda mundial, generando un desbalance en la balanza de pagos, mayor endeudamiento y aumento de inflación (20\%); aunque el país se recuperó por el descubrimiento de reservas petroleras, existió un alto optimismo sobre el futuro del país, el cual aumentó el endeudamiento con la premisa de un aumento en los precios del petroleo.
\end{itemize}


Uno de los obstáculos para el desarrollo económico en México hasta el día de hoy ha sido el descontento social, el cual genera problemáticas regionales-políticas que pueden crear inestabilidad, desde un sentido económico provoca la destrucción de capital reduciendo la acumulación del mismo, me parece que en México se ha construido con el tiempo una cultura de descontento social-político \cite{ReformasHistoria}.